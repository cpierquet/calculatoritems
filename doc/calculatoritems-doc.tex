% !TeX TXS-program:compile = txs:///arara
% arara: pdflatex: {shell: no, synctex: no, interaction: batchmode}
% arara: pdflatex: {shell: no, synctex: no, interaction: batchmode}

\documentclass[11pt,a4paper]{ltxdoc}
\usepackage{bera}
\usepackage{inconsolata}
\usepackage[T1]{fontenc}
\usepackage[utf8]{inputenc}
\usepackage[scale=0.875]{cabin}
\usepackage{calculatoritems}
\usepackage{fancyvrb}
\usepackage{fancyhdr}
\usepackage{tabularray}
\usepackage{fontawesome5}
\fancyhf{}
\renewcommand{\headrulewidth}{0pt}
\lfoot{\sffamily\small [calculatoritems]}
\cfoot{\sffamily\small - \thepage{} -}
\rfoot{\hyperlink{matoc}{\small\faArrowAltCircleUp[regular]}}
\usepackage{hologo}
\providecommand\tikzlogo{Ti\textit{k}Z}
\providecommand\TeXLive{\TeX{}Live\xspace}
\let\TikZ\tikzlogo

\usepackage{hyperref}
\urlstyle{same}
\hypersetup{pdfborder=0 0 0}
\usepackage[margin=2cm]{geometry}
\setlength{\parindent}{0pt}
\def\TPversion{0.1.0}
\def\TPdate{11/11/2024}
\usepackage{enumitem}
\usepackage{tcolorbox}
\usepackage{pgffor}
\tcbuselibrary{breakable,skins,hooks,listingsutf8}

\lstdefinestyle{packagestyle}
{
	language=[LaTeX]TeX,%
	columns=fullflexible,%
	extendedchars=true,%
	basicstyle=\small\ttfamily,%
	keywordstyle={\color{black}},%
	classoffset=0,%
	keywords={},%
	alsoletter={-},%
	keywordstyle={\color{blue}},%
	classoffset=1,%
	alsoletter={-},%
	morekeywords={},%
	keywordstyle={\color{violet}},%
	classoffset=2,%
	alsoletter={-},%
	morekeywords={calculatoritems,\CalcItemMenu,nwkstri,tidots,casiodots},%
	keywordstyle={\color{green!50!black}},%
	classoffset=3,%
	morekeywords={xelua,noamssymb,model,type,fsep,font,len,bg,rightsymb},%
	keywordstyle={\color{orange}},%
	inputencoding=utf8/latin1
}

\lstset{
%	language=[LaTeX]TeX,%
	basicstyle=\small\ttfamily,%
	keywordstyle={},%
%	classoffset=0,%
%	keywords={},%
%	alsoletter={-},%
%	keywordstyle={\color{blue}},%
%	classoffset=1,%
%	alsoletter={-},%
%	morekeywords={},%
%	keywordstyle={\color{violet}},%
%	classoffset=2,%
%	alsoletter={-},%
%	morekeywords={calculatoritems,\CalcItemMenu,nwkstri,tidots,casiodots},%
%	keywordstyle={\color{green!50!black}},%
%	classoffset=3,%
%	morekeywords={xelua,noamssymb,model,type,fsep,font,len,bg,rightsymb},%
%	keywordstyle={\color{orange}}
}

\newtcblisting{DemoCode}[1]{%
	enhanced,width=\linewidth,%
	bicolor,size=title,%
	colback=cyan!10!white,%
	colbacklower=cyan!5!white,%
	colframe=cyan!75!black,%
	listing options={%
		breaklines=true,%
		breakatwhitespace=true,%
		style=packagestyle,%
		basicstyle=\footnotesize\ttfamily,%
		tabsize=4,%
		commentstyle={\itshape\color{gray}},
		keywordstyle={\color{blue}},%
		classoffset=0,%
		keywords={\newfontfamily},%
		alsoletter={-},%
		keywordstyle={\color{blue}},%
		classoffset=1,%
		alsoletter={-},%
		morekeywords={\CalcItemMenu,\nwkstri,\tidots,\casiodots},%
		keywordstyle={\color{violet}},%
		classoffset=2,%
		alsoletter={-},%
		morekeywords={calculatoritems,\CalcItemMenu,nwkstri,tidots,casiodots},%
		keywordstyle={\color{green!50!black}},%
		classoffset=3,%
		morekeywords={xelua,noamssymb,model,type,fsep,font,len,bg,rightsymb},%
		keywordstyle={\color{orange}}
	},%
	#1
}

\newtcbinputlisting\DemoCodeFile[1]{%
	enhanced,width=\linewidth,%
	bicolor,size=title,%
	colback=lightgray!10!white,%
	colbacklower=lightgray!5!white,%
	colframe=lightgray!75!black,%
	listing options={%
		breaklines=true,%
		breakatwhitespace=true,%
		style=tcblatex,
		extendedchars=true,%
		basicstyle=\tiny\ttfamily,%
		keywordstyle={},%
		tabsize=2,%
		commentstyle={\itshape\color{gray}},%
		inputencoding=utf8/latin1
	},%
	breakable,
	listing only,%
	listing file={#1}
}

\NewDocumentCommand\ShowCode{ m }{%
	\colorbox{lightgray!50}{\lstinline!#1!}%
}

\begin{document}

\thispagestyle{empty}

\begin{center}
	\begin{minipage}{0.88\linewidth}
		\begin{tcolorbox}[colframe=yellow,colback=yellow!15]
			\begin{center}
				\renewcommand{\arraystretch}{1.25}%
				\begin{tabular}{c}
					{\Huge \texttt{calculatoritems}}\\
					\\
					{\LARGE Insert items of} \\
					{\LARGE classic calculators.} \\
					\\
					{\small \texttt{Version \TPversion{} -- \TPdate}}
				\end{tabular}
			\end{center}
		\end{tcolorbox}
	\end{minipage}
\end{center}

\begin{center}
	\begin{tabular}{c}
		\texttt{Cédric Pierquet}\\
		{\ttfamily c pierquet -- at -- outlook . fr}\\
		\texttt{\url{https://github.com/cpierquet/calculatoritems}} \\
	\end{tabular}
\end{center}

\hrule

\vfill

\begin{tcblisting}{colframe=lightgray,colback=lightgray!5,listing only}
Classic calculators items or menus :

35+E :
  \CalcItemMenu[model=35+,font=\fontCASIOA]{GRAPH}

90+E:
  \CalcItemMenu[model=90+,type=bmenu,font=\fontCASIOB]{MAT}

MATH+ :
  \CalcItemMenu[model=math+,font=\fontCASIOB,rightsymb=>]{arithmetic}

NWKS :
  \CalcItemMenu[model=nwks,type=bmenu,rightsymb=\nwkstri,len=12,font\fontNWKS]{X predict}

TI :
  \CalcItemMenu[model=ti,type=itemsel,font=\small\fontTI]{6§{fmin(}}

HP Prime :
  \CalcItemMenu[model=hp,type=itemsel,font=\small\fontHP,rightsymb=>]{4§Quadratic Explorer}
\end{tcblisting}

\begin{tcolorbox}[colframe=lightgray,colback=lightgray!5]
Classic calulators items or menus :

\begin{itemize}
	\item \texttt{35+E~} : \CalcItemMenu[model=35+,font=\fontCASIOA]{GRAPH}
	\item \texttt{90+E~} : \CalcItemMenu[model=90+,type=bmenu,font=\fontCASIOB,bg=lightgray!5]{MAT}
	\item \texttt{MATH+} : \CalcItemMenu[model=math+,font=\fontCASIOB,rightsymb=>]{arithmetic}
	\item \texttt{NWKS~} : \CalcItemMenu[model=nwks,type=bmenu,rightsymb=\nwkstri,len=14, font=\fontNWKS]{X predict}
	\item \texttt{TI~~~} : \CalcItemMenu[model=ti,type=itemsel,font=\fontTI]{6§{fmin(}}
	\item \texttt{HP~~~} : \CalcItemMenu[model=hp,type=itemsel,font=\fontHP,rightsymb=>]{4§Quadratic Explorer}
\end{itemize}
\end{tcolorbox}

\vfill~

%\hrule
%
%\medskip

%\emph{%
%	The \textsf{luarandom} package do the same things, but with the obligation to compile with \hologo{LuaLaTeX}.
%}
%
%\medskip

\hrule

\vspace*{5mm}

\pagebreak

\phantomsection

\hypertarget{matoc}{}

\tableofcontents

\vspace*{5mm}

\vfill

\section{History \& Future}

\texttt{0.1.0: Initial version}
\texttt{todo~: usage of nodepthtext package ?}

\vspace*{5mm}

%\hrule

\pagebreak

\section{Introduction}

\subsection{Loading, useful packages}

In order to load \ShowCode{calculatoritems}, simply use:

\begin{DemoCode}{listing only}
\usepackage{calculatoritems}
\end{DemoCode}

Loaded packages are \ShowCode{xstring}, \ShowCode{calc}, \ShowCode{simplekv}, \ShowCode{tcolorbox} and \ShowCode{circledtext}.

Loaded libraries are \ShowCode{calc} and \ShowCode{skins}.

\smallskip

If \ShowCode{ammsymb} doen't need to be loaded (useful for int. macro), just add \ShowCode{[noamssymb]} to the loading.

\begin{DemoCode}{listing only}
%w/o amssymb loading
\usepackage[noamssymb]{calculatoritems}
\end{DemoCode}

\subsection{Fonts}

The package define shortcuts for fonts, depending on the engine, an option \ShowCode{[xelua]} can be used.

\begin{DemoCode}{listing only}
%normal loading, for classic engines (pdflatex/latex)
\usepackage{calculatoritems}
\end{DemoCode}

\begin{DemoCode}{listing only}
%special loading, for recent engines (xelatex/lualatex)
\usepackage[xelua]{calculatoritems}
\end{DemoCode}

Available fonts are given by followings macros (best fonts are \texttt{teletype}).

\begin{DemoCode}{listing only}
%normal loading, for classic engines (pdflatex/latex)
\newcommand\fontNWKS{%
  \fontencoding{T1}\fontfamily{SourceCodePro-TLF}\selectfont %nwks
}
\newcommand\fontCASIOA{%
  \fontencoding{T1}\fontfamily{AnonymousPro}\fontseries{sb}\selectfont %casio35
}
\newcommand\fontCASIOB{%
  \fontencoding{T1}\fontfamily{AlegreyaSans-TLF}\fontseries{sb}\selectfont %casio90 & math+
}
\newcommand\fontTI{%
  \fontencoding{T1}\fontfamily{AnonymousPro}\fontseries{sb}\selectfont %ti
}
\newcommand\fontHP{%
  \fontencoding{T1}\fontfamily{AlegreyaSans-TLF}\fontseries{sb}\selectfont %hp
}
\end{DemoCode}

\begin{DemoCode}{listing only}
%special loading, for recent engines (xelatex/lualatex) with fontspec
\newfontfamily\fontNWKS{SourceCodePro-Medium} %numworks
\newfontfamily\fontCASIOA{AnonymousPro} %casio35
\newfontfamily\fontCASIOB{AlegreyaSans} %casio90 & math+
\newfontfamily\fontTI{AnonymousPro} %ti
\newfontfamily\fontHP{AlegreyaSans} %casio90 & math+
\end{DemoCode}

\subsection{Special macros}

Special macros are available, to match with some custom \textit{symbols}.

\begin{DemoCode}{}
\nwkstri \qquad \tidots  \qquad \casiodots
\end{DemoCode}

\section{Usage}

\subsection{Global usage}

The purpose of the main macro is to insert, \textit{inline}, a small \texttt{tcbox} to display \textit{items} as for classic calculators.

Size and aspect are fixed, in order to \textit{match} the original rendering.

\subsection{The macro}

The main macro is \ShowCode{\\CalcItemMenu}.

\begin{DemoCode}{listing only}
\CalcItemMenu[keys]{content}
\end{DemoCode}

Available keys are :

\begin{itemize}[leftmargin=*]
	\item \ShowCode{model} : specify the model (\texttt{empty} by default) ;
	\item \ShowCode{type} : type of item, according to the specified model (\texttt{empty} by default) ;
	\item \ShowCode{fsep} : length for modifying the sep between rules and content (\texttt{1pt} by default) ;
	\item \ShowCode{font} : font for the content (\texttt{\textbackslash bfseries\textbackslash ttfamily} by default) ;
	\item \ShowCode{len} : internal key for modifying length of content, for same models/types (\texttt{auto} by default) ;
	\item \ShowCode{bg} : bg color or the \textit{external background}, if necessary (\texttt{white} by default) ;
	\item \ShowCode{rightsymb} : right symbol, if necessary (\texttt{empty} by default).
\end{itemize}

\section{Samples}

\subsection{Generic model}

This is the default rendering.

Available items are :

\begin{itemize}[leftmargin=*]
	\item \ShowCode{[type=\{\}]} := white menu (default value)\hfill\CalcItemMenu{MyItem}
	\item \ShowCode{[type=black]} := black menu\hfill\CalcItemMenu[type=black]{MyItem}
\end{itemize}

\begin{DemoCode}{listing only}%
\CalcItemMenu{MyItem}
\CalcItemMenu[type=black]{MyItem}
\end{DemoCode}

\subsection{CASIO 35+ or fx-9860GIII}

For this model, the key is \ShowCode{[model=35+]}, and font \ShowCode{[font=\\fontCASIOA]} can be used.

By default, there's 4 \textit{characters} in the box, so if there's more, a \textit{h-stretch} is applied.

Available items are :

\begin{itemize}[leftmargin=*]
	\item \ShowCode{[type=\{\}]} := white menu (default value) \hfill\CalcItemMenu[model=35+,font=\small\fontCASIOA]{GRPH}
	\item \ShowCode{[type=bmenu]} := dark menu \hfill\CalcItemMenu[model=35+,type=bmenu,font=\small\fontCASIOA]{GRPH}
	\item \ShowCode{[type=item]} := item menu \hfill\CalcItemMenu[model=35+,type=item,font=\small\fontCASIOA]{GRPH}
	\item \ShowCode{[type=itemsel]} := item selected (19 chars) with optional right symbol\hfill\CalcItemMenu[model=35+,type=itemsel,font=\small\fontCASIOA]{TEST LONG ITEM}
\end{itemize}

\begin{DemoCode}{listing only}
\CalcItemMenu[model=35+,font=\small\fontCASIOA]{GRPH}
\CalcItemMenu[model=35+,type=bmenu,font=\small\fontCASIOA]{GRPH}
\CalcItemMenu[model=35+,type=item,font=\small\fontCASIOA]{GRPH}
\CalcItemMenu[model=35+,type=itemsel,font=\small\fontCASIOA]{TEST LONG ITEM}
\end{DemoCode}

\subsection{CASIO 90+ or fx-CG50}

For this model, the key is \ShowCode{[model=90+]}, and font \ShowCode{[font=\\fontCASIOB]} can be used.

By default, there's 5 \textit{characters} in the box, so if there's more, a \textit{h-stretch} is applied.

Available items are :

\begin{itemize}[leftmargin=*]
	\item \ShowCode{[type=\{\}]} := white menu (default value) \hfill\CalcItemMenu[model=90+,font=\small\fontCASIOB]{GRAPH}
	\item \ShowCode{[type=bmenu]} := black menu \hfill\CalcItemMenu[model=90+,type=bmenu,font=\small\fontCASIOB]{GRAPH}
	\item \ShowCode{[type=item]} := item menu \hfill\CalcItemMenu[model=90+,type=item,font=\small\fontCASIOB]{GRAPH}
	\item \ShowCode{[type=itemsel]} := item selected (22 chars) with optional right symbol
	
	\hfill\CalcItemMenu[model=90+,type=itemsel,font=\small\fontCASIOB]{TEST LONG ITEM}
\end{itemize}

\begin{DemoCode}{listing only}
\CalcItemMenu[model=90+,font=\small\fontCASIOB]{GRAPH}
\CalcItemMenu[model=90+,type=bmenu,font=\small\fontCASIOB]{GRAPH}
\CalcItemMenu[model=90+,type=item,font=\small\fontCASIOB]{GRAPH}
\CalcItemMenu[model=90+,type=itemsel,font=\small\fontCASIOB]{TEST LONG ITEM}
\end{DemoCode}

\subsection{CASIO MATH+}

For this model, the key is \ShowCode{[model=math+]} (20 chars), and font \ShowCode{[font=\\fontCASIOB]} can be used.

Only one item is available, due to \textit{new global usage}, but \texttt{rightsymb} can be used.

\begin{itemize}[leftmargin=*]
	\item \ShowCode{[rightsymb=\{\}]} (default)\hfill\CalcItemMenu[model=math+,font=\small\fontCASIOB]{MyItem}
	\item \ShowCode{[rightsymb=>]}\hfill\CalcItemMenu[model=math+,font=\small\fontCASIOB,rightsymb=>]{MyItem}
	\item \ShowCode{[rightsymb=\\casiodots]}\hfill\CalcItemMenu[model=math+,font=\small\fontCASIOB,rightsymb=\casiodots]{MyItem}
\end{itemize}

\begin{DemoCode}{listing only}
\CalcItemMenu[model=math+,font=\small\fontCASIOB]{MyItem}
\CalcItemMenu[model=math+,font=\small\fontCASIOB,rightsymb=>]{MyItem}
\CalcItemMenu[model=math+,font=\small\fontCASIOB,rightsymb=\casiodots]{MyItem}
\end{DemoCode}

\subsection{NUMWORKS}

For this model, the key is \ShowCode{[model=nwks]}, and font \ShowCode{[font=\\fontNWKS]} can be used.

Available items are :

\begin{itemize}[leftmargin=*]
	\item \ShowCode{[type=\{\}]} := white menu (default)\hfill\CalcItemMenu[model=nwks,font=\small\fontNWKS]{MyItem}
	\item \ShowCode{[type=gmenu]} := gray menu\hfill\CalcItemMenu[model=nwks,type=gmenu,font=\small\fontNWKS]{MyItem}
	\item \ShowCode{[type=bmenu]} := black menu (22 chars, with \texttt{rightsymb})\hfill\CalcItemMenu[model=nwks,type=bmenu,font=\small\fontNWKS,rightsymb=\nwkstri]{MyItem}
\end{itemize}

\begin{DemoCode}{listing only}
\CalcItemMenu[model=nwks,font=\small\fontNWKS]{MyItem}
\CalcItemMenu[model=nwks,type=gmenu,font=\small\fontNWKS]{MyItem}
\CalcItemMenu[model=nwks,type=bmenu,font=\small\fontNWKS,rightsymb=\nwkstri]{MyItem}
\end{DemoCode}

\subsection{TI}

For this model, the key is \ShowCode{[model=ti]}, and font \ShowCode{[font=\\fontTI]} can be used.

Available items are :

\begin{itemize}[leftmargin=*]
	\item \ShowCode{[type=\{\}]} := black menu (default)\hfill\CalcItemMenu[model=ti,font=\small\fontTI]{MyItem}
	\item \ShowCode{[type=menu]} := default menu\hfill\CalcItemMenu[model=ti,type=menu,font=\small\fontTI]{MyItem}
	\item \ShowCode{[type=itemsel]} := selected itemn, with number\hfill\CalcItemMenu[model=ti,type=itemsel,font=\small\fontTI]{1§{MyItem\tidots}}
\end{itemize}

\begin{DemoCode}{listing only}
\CalcItemMenu[model=ti,font=\small\fontTI]{MyItem}
\CalcItemMenu[model=ti,type=menu,font=\small\fontTI]{MyItem}
\CalcItemMenu[model=ti,type=itemsel,font=\small\fontTI]{1§{MyItem\tidots}}
\end{DemoCode}

\subsection{HP Prime}

For this model, the key is \ShowCode{[model=hp]}, and font \ShowCode{[font=\\fontHP]} can be used.

By default, there's 5 \textit{characters} in the box, so if there's more, a \textit{h-stretch} is applied.

Available items are :

\begin{itemize}[leftmargin=*]
	\item \ShowCode{[type=\{\}]} := semi-rounded (default value) \hfill\CalcItemMenu[model=hp,font=\small\fontHP]{Catlg}
	\item \ShowCode{[type=ritem]} := rounded \hfill\CalcItemMenu[model=hp,type=ritem,font=\small\fontHP]{OK}
	\item \ShowCode{[type=item]} := item with optional right symbol\hfill\CalcItemMenu[model=hp,type=item,font=\small\fontHP,rightsymb={~>}]{1§Extremum}
	\item \ShowCode{[type=itemsel]} := item selected (21 chars) with optional right symbol
	
	\hfill\CalcItemMenu[model=hp,type=itemsel,font=\small\fontHP,rightsymb=>]{4§Quadratic Explorer}
\end{itemize}

\begin{DemoCode}{listing only}
\CalcItemMenu[model=hp,font=\small\fontHP]{Catlg}
\CalcItemMenu[model=hp,type=ritem,font=\small\fontHP]{OK}
\CalcItemMenu[model=hp,type=item,font=\small\fontHP,rightsymb={~>}]{1§Extremum}
\CalcItemMenu[model=hp,type=itemsel,font=\small\fontHP,rightsymb=>]{4§Quadratic Explorer}
\end{DemoCode}


\pagebreak

\section{The code}

\DemoCodeFile{calculatoritems.sty}

\end{document}
